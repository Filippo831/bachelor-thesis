\chapter{Progetto}
\label{cha:Progetto}

\section{Descrizione}
Il progetto consiste è nell'implementazione di un reverse proxy nel linguaggio di programmazione go e successivamente containerizzato tramite docker. Il reverse proxy sviluppato effettua alcune delle funzionalità basi necessarie per la comunicazione. Le funzionalitá implementate sono quindi quelle di routing, supporto a https e wss tramite TLS, e la possibilitá di configurarlo tramite file json.

\section{Obiettivi}
\subsection{Implementativi}
L'obiettivo principale di questo progetto era quello di creare un software che riuscisse a funzionare con comunicazioni http. La lista degli obiettivi inizialmente quindi era:
\begin{enumerate}
  \item supporto http
  \item configurazione via file
  \item sistema di logging per il controllo di eventuali errori
\end{enumerate}
Durante l'implementazione però ho visto che go ha delle librerie molto ben fornite per lo sviluppo di server e comunque della gestione dei protocolli di comunicazione. Quando ho visto la possibilità di ampliare la lista degli obiettivi cercando di includere anche tecnologie nuove come http3. La lista degli obiettivi finali quindi è diventata:
\begin{enumerate}
  \item supporto http/https
  \item supporto websocket (ws/wss)
  \item configurazione via file
  \item chunk encoding
  \item supporto dei subdomain
  \item supporto http2
  \item supporto http3 (anche se non perfettamente funzionante)
  \item sistema di logging per il controllo di eventuali errori
\end{enumerate}

\subsection{Conoscenze}
L'obiettivo principale che ha fatto nascere questo progetto era l'interesse mio nell'approfondire e mettere in pratica le nozioni apprese in materia di comunicazioni web nei corsi che ho frequentato. Anche la scelta del linguaggio di programmazione go è stata fatta con lo scopo di imparare un linguaggio di programmazione che sta diventando via via sempre piú popolare nello sviluppo di server o applicativi che hanno a che fare con il networking. Questo grazie alla sua natura di basso livello che lo rende molto veloce nell'esecuzione ma anche alla vasta varietà di librerie standard che velocizzano molto lo sviluppo, lasciando spazio all'implementazione di nuove feature. Questo sarà anche un punto a favore nel mio progetto perché mi ha lasciato spazio per testare e provare piú funzionalitá di quelle che avevo previsto.

\section{Struttura}
Il progetto é stato strutturato con una struttura base ampliamente utilizzata nei progetti in go che consiste nel dividere i file di programmazione in file interni o file eseguibili. Nei file interni ci sono tutte le funzioni di supporto che poi verranno utilizzate per far funzionare i comandi eseguibili.
\subsection{Schema ad albero}
\begin{verbatim}
- cmd
   - reverse_proxy
- configuration.json
- Dockerfile
- go.mod
- go.sum
- internal
   - http_handler
   - read_configuration
   - reverse_proxy
   - run_server
   - websocket_handler
- test
\end{verbatim}
\subsection{descrizione elementi}
\begin{itemize}[label={}]
  \item \textbf{cmd}: cartella dove vengono inseriti i file eseguibili, in questo caso l'unico comando eseguibile é \texttt{reverse\_proxy}
  \item \textbf{configuration.json}: nel file di configurazione vengono definiti i server da abilitare, le caratteristiche del server e tutte le impostazioni di routing.
  \item \textbf{Dockerfile}: file di configurazione del container docker
  \item \textbf{go.mod, go.sum}: file di configurazione dell'ambiente di sviluppo go.
  \item \textbf{internal}: cartella in cui vengono definite le funzioni che verranno eseguite dal reverse proxy.
  \item \textbf{test}: cartella con tutti i file di test
\end{itemize}

\section{Configurazione}

\subsection{Struttura file configurazione}
\begin{lstlisting}[language=Javascript]
  {
	"servers": [
    	{
        	"port": int,
        	"server_name": string,
        	"http3": bool,
        	"ssl_to_client": bool,
        	"ssl_certificate": string,
        	"ssl_certificate_key": string,
        	"max_redirect": int,
        	"chunk_encoding": bool,
        	"chunk_size": int,
        	"chunk_timeout": int,
        	"location": [
            	{
                	"subdomain": string,
                	"to": string
            	},
        	]
    	}
	]
}
\end{lstlisting}
\subsection{Descrizione}
\begin{itemize}[label={}]
  \item \textbf{servers}: Lista dei server che si vogliono avviare all'avviamento del programma.
  \item \textbf{port}: Parta di accesso al server.
  \item \textbf{server\_name}: dominio del server.
  \item \textbf{http3}: Abilitare o no il server anche in modalitá http3.
  \item \textbf{ssl\_to\_client}: Abilitare o no la criptazione dei messaggi da reverse proxy a client.
  \item \textbf{ssl\_certificate}: Percorso del certificato ssl.
  \item \textbf{ssl\_certificate\_key}: Percorso della chiave del certificato.
  \item \textbf{chunk\_encoding}: Impostare l'invio dei dati in pacchetti per diminuire la dimensione di ognuno.
  \item \textbf{chunk\_size}: grandezza del pacchetto.
  \item \textbf{chunk\_timeout}: tempo entro la quale il pacchetto viene inviato anche se non ha raggiunto la dimensione definita da \texttt{chunk\_size}.
  \item \textbf{location}: Lista delle direttive di routing in base al sottodominio.
  \item \textbf{subdomain}: Indica il sottodominio che poi viene collegato al server dietro a \texttt{to}.
  \item \textbf{to}: Locazione del server a cui vengono inviate le richieste mandate tramite il sottodominio definito precedentemente.
\end{itemize}

\section{Implementazione}
\subsection{Lettura file configurazione}
Essendo il file di configurazione un json, si possono sfruttare le librerie di go per generare una struttura con la stessa struttura definita nella configurazione partendo dal file json. Viene quindi definita la struttura indicando ogni variabile a quale chiave corrisponde.
\begin{lstlisting}[language=Golang]
type Configuration struct {
	Http []Server `json:"servers"`
}
type Server struct {
	Port int `json:"port"`
	ServerName string `json:"server_name"`
	Http3Active bool `json:"http3"`
	Location []Location `json:"location"`
	SslToClient bool `json:"ssl_to_client"`
	SslCertificate string `json:"ssl_certificate"`
	SslCertificateKey string `json:"ssl_certificate_key"`
	MaxRedirect int `json:"max_redirect"`
	ChunkEncoding bool `json:"chunk_encoding"`
	ChunkSize int `json:"chunk_size"`
	ChunkTimeout int `json:"chunk_timeout"`
}

type Location struct {
	Domain string `json:"domain"`
	To string `json:"to"`
}
\end{lstlisting}
Questa struttura viene poi popolata con le informazioni presenti nel file con questa sequenza di funzioni.
\begin{lstlisting}[language=Golang]
var Conf Configuration
jsonFile, err := os.Open(filePath)
byteValue, err := io.ReadAll(jsonFile)
readingJsonErr := json.Unmarshal(byteValue, &Conf)
\end{lstlisting}
Successivamente vengono fatti dei test per effettuare queste verifiche:
\begin{itemize}
  \item Le impostazioni per abilitare la criptografia sono state inserite correttamente.
  \item Non ci sono sottodomini nello stesso server che si sovrappongono.
  \item I valori indicati nel \texttt{chunk\_encoding} non eccedono i limiti.
\end{itemize}

\subsection{Avvio servers}
Per l'avvio dei server definiti all'interno della configurazione, viene utilizzato un ciclo for che prende ogni server e lo utilizza per creare la configurazione di avvio.
\begin{lstlisting}[language=Golang]
for _, server := range readconfiguration.Conf.Http {
\end{lstlisting}
All'interno viene poi definita la funzione di Handler che contiene le funzioni da eseguire quando arriva una richiesta.
\begin{lstlisting}[language=Golang]
  proxy := http.HandlerFunc(func(w http.ResponseWriter, r *http.Request) {
	...
  }
\end{lstlisting}
Successivamente in base alle caratteristiche del server, quindi se il supporto a http3 è attivo oppure se l'SSL è attiva, viene chiamata una funzione diversa dalla libreria http di go che avvia il server desiderato. Per far funzionare più server in contemporanea, ognuno viene fatto eseguire dentro un thread dedicato.

\begin{lstlisting}[language=Golang]
  var wg sync.WaitGroup

  if server.Http3Active {
	wg.Add(2)
	go func() {
  	defer wg.Done()
  	errs <- http2Server.ListenAndServeTLS(server.SslCertificate, server.SslCertificateKey)
	}()

	go func() {
  	defer wg.Done()
  	errs <- http3Server.ListenAndServeTLS(server.SslCertificate, server.SslCertificateKey)
	}()

  } else {
	wg.Add(1);
    	if sslCertificate == "" && sslKey == "" && !sslActivate {
    	RunHttp(proxy, port, readTimeout, writeTimeout, idleTimeout, wg)
	} else if sslCertificate != "" && sslKey != "" && sslActivate {
    	RunHttps(proxy, port, readTimeout, writeTimeout, idleTimeout, sslCertificate, sslKey, wg)
	}
  }
\end{lstlisting}
\subsubsection{http3}
Quando viene selezionato il server http3, viene comunque avviato anche un server http2 di supporto. Questo serve perché la prima connessione da parte di un client viene sempre fatta tramite il protocollo http2. Utilizzando http2 tcp mentre http3 udp, quest'ultimo non puó rispondere alla richiesta http2. Il server http2 quindi risponde alla prima richiesta notificando che si puó cambiare la comunicazione ed elevarla al protocollo http3. Per notificare la presenza del server http3 si aggiunge questa stringa nell'Header del messaggio.\\
\texttt{alt-svc h3=":\$PORT"; ma=86400}

\subsection{Funzione Handler}
Questa funzione è quella che gestisce le comunicazioni che arrivano dai vari client. Quindi ha il compito di inoltrare la richiesta al server desiderato creando una richiesta uguale a quella ricevuta e modificando il destinatario.\\La prima operazione che viene effettuata é la verifica del subdomain indicato per vedere se questo effettivamente esiste nei file di configurazione.
\begin{lstlisting}[language=Golang]
proxy := http.HandlerFunc(func(w http.ResponseWriter, r *http.Request) {
  redirectURL, _ := nil
  found := false
  domain := r.Host
  domain = strings.Split(domain, ":")[0]

  for _, location := range server.Location {
	if domain == location.Domain {
  	redirectURL, _ = url.Parse(location.To)
  	found = true
	}
  }
  if !found {
	fmt.Fprintf(w, "subdomain not found")
	return
  }
\end{lstlisting}

Una volta trovato l'indirizzo del server destinatario si puó creare la richiesta da inviare.
\begin{lstlisting}[language=Golang]
	r.Host = redirectURL.Host
	r.URL.Host = redirectURL.Host
	r.URL.Scheme = redirectURL.Scheme
	r.RequestURI = ""

	client_host, _, _ := net.SplitHostPort(r.RemoteAddr)
	r.Header.Set("X-Forwarded-For", client_host)
\end{lstlisting}
E a questo punto la richiesta viene inviata al server e la risposta ricevuta viene inoltrata al client. Il processo a questo punto viene differenziato in base che la connessione sia tramite websocket oppure http.
\begin{lstlisting}[language=Golang]
  if r.Header.Get("Upgrade") == "websocket" {
	websocket_handler.Handle_websocket(w, redirect, server.SslToClient, redirectURL.Scheme == "https")
  } else {
	http_handler.HttpHandler(w, redirect, server)
  }
\end{lstlisting}


\section{Containerizzazione}
Alla fine per creare il container bisogna scrivere un file \texttt{Dockerfile} con all'interno le direttive sui file da includere, i comandi da eseguire e altre informazioni utili all'esecuzione.
\subsection{Analisi file}
\begin{lstlisting}[language=Dockerfile]
  FROM golang:1.24-alpine AS builder
\end{lstlisting}
Definizione dell'ambiente di compilazione del reverse proxy. Viene utilizzata la versione go per linux alpine perché verrá poi eseguito su quest'ultimo. Linux alpine é stato scelto per la ridotta dimensione che rendeva la dimensione dell'immagine ridotta.
\\
\begin{lstlisting}[language=Dockerfile]
  WORKDIR /app
  COPY go.mod go.sum ./
  RUN go mod download
\end{lstlisting}
Creazione della cartella \texttt{/app} all'interno del container. Tutti i file copiati successivamente andranno a posizionarsi all'interno di questa cartella. Vengono poi copiati i file contenenti le informazioni e le librerie necessarie per eseguire il codice. Una volta copiati questi 2 file viene chiamata la funzione di go per scaricare tutti i pacchetti necessari.
\\
\begin{lstlisting}[language=Dockerfile]
  COPY cmd/ ./cmd/
  COPY internal/ ./internal/
\end{lstlisting}
Copia delle cartelle contenenti tutti i file \texttt{*.go}.
\\
\begin{lstlisting}[language=Dockerfile]
  RUN go build -o /reverse_proxy ./cmd/reverse_proxy
\end{lstlisting}
Compilazione del progetto e creazione del file eseguibile.
\\
\begin{lstlisting}[language=Dockerfile]
  FROM alpine:latest
  COPY --from=builder /reverse_proxy /reverse_proxy
\end{lstlisting}
Copia del file eseguibile in un nuovo ambiente, questa volta non contenente i pacchetti go, cosí da essere ancora piú leggero.
\\
\begin{lstlisting}[language=Dockerfile]
  EXPOSE 8081
  EXPOSE 8082
\end{lstlisting}
Esposizione delle porte 8081 e 8082.
\\
\begin{lstlisting}[language=Dockerfile]
  ENTRYPOINT ["/reverse_proxy"]
\end{lstlisting}
Assegnazione del comando da eseguire all'avvio del container. In questo caso viene avviato il reverse proxy.

