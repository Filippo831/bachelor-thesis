\chapter{Sperimentale}
\label{cha:Sperimentale}

\section{Ambiente}
Per effettuare i test delle performance del reverse proxy é stato utilizzato un server sperimentale eseguito all'interno di un container docker. Per avviare tutti i servizi necessari per effettuare i test é stato creato un docker compose con le direttive per avviare il reverse proxy e il server di test.

\begin{lstlisting}[language=DockerCompose]
version: "1.0"
services:
  web:
    build: .
    ports:
      - "8081:8081"
      - "8082:8082"
    volumes:
      - "./configuration.json:/configuration_docker.json"
      - "./reverse_proxy.com+3.pem:/reverse_proxy.com+3.pem"
      - "./reverse_proxy.com+3-key.pem:/reverse_proxy.com+3-key.pem"
      - "./log_file.log:/log_file.log"

  testserver:
    image: "kennethreitz/httpbin"
    ports:
      - "8080:80"
\end{lstlisting}

\subsection{httpbin}
\cite{httpbin}
Httpbin é un server con delle chiamate base che possono essere utilizzate per testare i propri dispositivi. Si possono quindi testare chiamate semplici come semplici \texttt{GET} e \texttt{POST} ma anche alcune piú complesse come chiamate per verificare il funzionamento dei cookies oppure per verificare i redirect. Di questo servizio esiste giá la versione containerizzata nello store di docker che quindi ha reso l'utilizzo immediato.
