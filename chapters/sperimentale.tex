\chapter{Sperimentale}
\label{cha:Sperimentale}

\section{Ambiente}
\subsection{rete docker}
Per effettuare i test delle performance del reverse proxy é stato utilizzato un server sperimentale eseguito all'interno di un container docker. Per avviare tutti i servizi necessari per effettuare i test é stato creato un docker compose con le direttive per avviare il reverse proxy e il server di test.

\begin{lstlisting}[language=DockerCompose]
version: "1.0"
services:
  web:
    build: .
    ports:
      - "8081:8081"
      - "8082:8082"
    volumes:
      - "./configuration_docker.json:/configuration.json"
      - "./reverse_proxy.com+3.pem:/reverse_proxy.com+3.pem"
      - "./reverse_proxy.com+3-key.pem:/reverse_proxy.com+3-key.pem"
      - "./log_file.log:/log_file.log"

  testserver:
    image: "kennethreitz/httpbin"
    ports:
      - "8080:80"
\end{lstlisting}

\subsection{httpbin}
\cite{httpbin}
Httpbin é un server con delle chiamate base che possono essere utilizzate per testare i propri dispositivi. Si possono quindi testare chiamate semplici come semplici \texttt{GET} e \texttt{POST} ma anche alcune piú complesse come chiamate per verificare il funzionamento dei cookies oppure per verificare i redirect. Di questo servizio esiste giá la versione containerizzata nello store di docker che quindi ha reso l'utilizzo immediato.

\subsection{jmeter}
\cite{jmeter}
Jmeter é un software principalmente utilizzato per testare servizi che si basano su protocolli di communicazione di quinto livello. Quindi si possono testare API tramite il protocollo http o https, ma anche condivisione mail con POP3 o IMAP e altri come FTP.

\subsection{wireshark}
\cite{wireshark}
Wireshark é un software per analizzare il traffico passante per un porta di rete. Una volta dato l'accesso alla porta direte, wireshark mostra tutti i pacchetti tcp/udp che passano per quella porta. Quindi in questo caso si parla di protocolli di quarto livello.

\section{Elenco dei test}\label{test:settings}
Per verificare le performance del reverse proxy e valutare le differenze rispetto a una connessione diretta al servizio, sono stati effettuati test in quattro situazioni differenti.
\begin{enumerate}
  \item
    \begin{itemize}
      \item Reverse proxy: non collegato
      \item server: connesso tramite \texttt{http}
    \end{itemize}
  \item
    \begin{itemize}
      \item Reverse proxy: connesso tramite \texttt{http}
      \item server: connesso tramite \texttt{http}
    \end{itemize}
  \item
    \begin{itemize}
      \item Reverse proxy: connesso tramite \texttt{https}
      \item server: connesso tramite \texttt{http}
    \end{itemize}
\end{enumerate}
I fattori interessanti da valutare analizzando i risultati di questi test sono:
\begin{itemize}
  \item Quanta differeza di velocitá e grandezza dei pacchetti c'é tra una connessione non criptata e una criptata.
  \item Quanto differisce la velocitá di accesso alle risorse del server tra la connessione diretta e l'utilizzo di un reverse proxy.
\end{itemize}

\section{Procedura esecuzione test}
Per ognuna delle configurazioni definite precedentemente\ref{test:settings} vengono fatte 2 tipologie di test.
\begin{enumerate}
  \item Tramite wireshark viene tracciata una richiesta al server per poi verificare la quantitá di pacchetti inviati, la loro dimensione totale e il tempo che ci mette ad ultimare.
  \item Tramite jmeter viene fatto un test a carico, cioé generando molte richieste e tenendo traccia del tempo impiegato per ultimarle e la dimensione totale dei pacchetti inviati. In questo caso vengono fatte 1000 richieste per ogni test.

\end{enumerate}

\section{Risultati}
\subsection{Singola connessione}
In tabella sono riportati la quantitá di pacchetti scambiati e la dimensione totale di questi per una singola richiesta.
\begin{center}
  \begin{tabular}{|c|c|c|}
    \hline
    condizione & numero pacchetti & dimensione totale (bytes) \\
    \hline
    \hline
    http server & 14 & 9906 \\
    \hline
    http reverse proxy + http server & 14 + 12 = 26 & 9908 + 9961 = 19869 \\
    \hline
    https reverse proxy + http server & 27 + 14 = 41 & 12000 + 9961 = 21961 \\
    \hline

  \end{tabular}
\end{center}
  \subsection{1000 connessioni}
  In tabella é riportato il tempo impiegato per effettuare 1000 connessioni in sequenza da signolo thread.
  \begin{center}
    \begin{tabular}{|c|c|c|c|}
      \hline
      condizione & risultati (s) & media (s) & variazione standart (s) \\
      \hline
      \hline
      http server & \makecell {2.99 \\ 2.83 \\ 2.87} & 2.896 & 0.083 \\
      \hline
      http reverse proxy + http server & \makecell {4.00 \\ 3.95 \\ 3.92} & 3.956 & 0.040 \\
      \hline
      https reverse proxy + http server & \makecell {4.28 \\ 3.93 \\ 3.96} & 4,057 & 0,194 \\
      \hline
    \end{tabular}

  \end{center}

